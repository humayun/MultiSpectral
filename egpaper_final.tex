\documentclass[10pt,twocolumn,letterpaper]{article}

\usepackage{iccv}
\usepackage{times}
\usepackage{epsfig}
\usepackage{graphicx}
\usepackage{amsmath}
\usepackage{amssymb}

% Include other packages here, before hyperref.

% If you comment hyperref and then uncomment it, you should delete
% egpaper.aux before re-running latex.  (Or just hit 'q' on the first latex
% run, let it finish, and you should be clear).
\usepackage[breaklinks=true,bookmarks=false]{hyperref}

\iccvfinalcopy % *** Uncomment this line for the final submission

\def\iccvPaperID{****} % *** Enter the ICCV Paper ID here
\def\httilde{\mbox{\tt\raisebox{-.5ex}{\symbol{126}}}}

% Pages are numbered in submission mode, and unnumbered in camera-ready
%\ificcvfinal\pagestyle{empty}\fi
\setcounter{page}{4321}
\begin{document}

%%%%%%%%% TITLE
\title{Multispectral Texture characterization: Application to Mitotic Count in Breast Cancer Histopathology}

\author{First Author\\
Institution1\\
Institution1 address\\
{\tt\small firstauthor@i1.org}
% For a paper whose authors are all at the same institution,
% omit the following lines up until the closing ``}''.
% Additional authors and addresses can be added with ``\and'',
% just like the second author.
% To save space, use either the email address or home page, not both
\and
Second Author\\
Institution2\\
First line of institution2 address\\
{\tt\small secondauthor@i2.org}
}

\maketitle
%\thispagestyle{empty}

%%%%%%%%% ABSTRACT
\begin{abstract}
Multispectral Imaging (MSI) ...
\end{abstract}

%%%%%%%%% BODY TEXT
\section{Introduction}
%Medical (Scanner, big objects, not to many) vs Biology/biomedical (microscopic, small objects, large in quantity)

In medical imaging, computer aided detection (CADe), also called computer aided diagnosis (CADx) systems have played a significant part in medicine that assist doctors in interpretation of medical images. CAD systems are extensively used in the detection and differential diagnosis of many different types of abnormalities in medical images obtained using different imaging modalities. Medical imaging can be divided into radiology and pathology. Both the imaging modalities look for anatomical or physiological reasons for injury or disease. A radiologist is trained to review x-ray, and other non-invasive images and scans of the anatomical object like brain, lung, heart (e.g., ultrasound or MRI) for diagnosis. A large amount of research are available on CAD in radiology including different image modalities like x-rays, ultrasound, MRI, CT and mammogram ~\cite{sonka2000,bankman2000,erickson2002,summers2003,giger2004}. The pathologist is also trained to perform such evaluation on biological objects, but with actual tissue samples. The standard computer vision methodology are employed in mostly CAD systems involved object detection, segmentation and classification. However, in the recent years, CAD systems in pathology is an active area of research and posing more challenging problem as compared to radiology where pathologists traditionally make a diagnostic decision by viewing a specimen under microscope and measuring various diagnostically important attributes of an isolated object such as size, shape, darkness, color and texture.

Image processing and computer vision techniques are of special interest for pathologists because they can be more reliable than classical assessment by eye screening using poorly defined criteria. Being important in diagnostic pathology, this qualitative assessment is also used to understand the ground realities for specific diagnostic being rendered like specific chromatin texture in cancerous nuclei, which may indicate certain genetic abnormalities. In addition, quantitative characterization of pathology imagery is important not only for clinical applications (e.g., to reduce/eliminate inter- and intra-observer variation in diagnosis) but also for research applications (e.g., to understand the biological mechanisms of the disease process~\cite{gurcan2009}. 

Histopathological study has been conducted for numerous cancer detection and grading applications including brain~\cite{ alkadi2010, sertel2009}, breast~\cite{ petushi2006,dundar2011,basavanhally2012},cervix~\cite{ Keenan2000} etc. One of the most difficult fields in histopathology imagery is spatial analysis, more specifically automated nuclei detection and classification. The objective of nuclei classification is assign label to different type of nuclei as normal, cancer, mitotic, apoptosis, lymphocytes etc that in particular, a challenging problem to address in histopathology. H\&E is a well established staining technique in histopathology that exploits intensity of stains in the tissue images to quantify the nuclei and other structures related to cancer developments. Image processing techniques in this context are devoted to the accurate and objective quantification and localization of such activity in specific regions of the tissue such as cytoplasm, membranes and nuclei. During the last decade, a large amount of research has focused on histology imagery, which raised much challenging imaging problems than with cytology imagery. There are several strategies for nuclei classification in color (RGB) histological images that lack in accuracy due to the process of acquisition (slicing, staining, teering, etc) creates poor 3D correlation in image color, gradient and structures.

Pathology provides the critical link between the biological basis of an image or spectral signature and clinical outcomes obtained through optical imaging. The validation of optical images and spectra requires both morphologic diagnosis from histopathology and parametric analysis of tissue features above and beyond the declared pathologic diagnosis ~\cite{wells2007}. From the chromatic viewpoint, nuclear regions are characterized by non-uniform stain intensity and color, thus preventing a trivial classification based on color separation. In fact, the superposition of tissue layers as well as the diffusion of the dyes on the tissue surface may bring the stains to contaminate the background or other cellular regions which are different from their specific target. Moreover, different portions of the same tissue are may be not equally enlightened and stained. Indeed, multispectral images will allow biologists and pathologists to see beyond the RGB image planes that they are accustomed to. 

Multispectral imaging (MSI) is currently in a period of transition from its role as an exotic technique to its being offered in one form or another by all the major microscopy manufacturers. This is because it provides solutions to some of the major challenges in fluorescence-based imaging, namely ameliorating the consequences of the presence of autofluorescence and the need to easily accommodate relatively high levels of signal multiplexing. MSI, which spectrally characterizes and computationally eliminates autofluorescence, enhances the signal-to-background dramatically, revealing otherwise obscured targets. While this article concentrates on detection of mitotic nuclei in multispectral images, the intent is to showcase the advantages of multispectral imaging in general. 

%Softwares (MicroMSI, Multispec, Gerbil)
%-------------------------------------------------------------------------
\section{Literature Review}


%------------------------------------------------------------------------
\section{Proposed Method}

\begin{figure*}
\begin{center}
\fbox{\rule{0pt}{2in} \rule{.9\linewidth}{0pt}}
\end{center}
   \caption{Example of a short caption, which should be centered.}
\label{fig:short}
\end{figure*}

%------------------------------------------------------------------------
\section{Experiments and Results}

%------------------------------------------------------------------------
\section{Discussion and Conclusion}

%------------------------------------------------------------------------
{\small
\bibliographystyle{ieee}
\bibliography{egbib}
}
\end{document}
